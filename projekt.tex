\documentclass[11pt]{article}
 
\usepackage[T1]{fontenc}
\usepackage[polish]{babel}
\usepackage[utf8]{inputenc}
\usepackage{lmodern}
\selectlanguage{polish}


\begin{document}

%----------------------------------------------------------------------------------------
%	Strona Tytułowa
%----------------------------------------------------------------------------------------

\begin{titlepage}

	\newcommand{\HRule}{\rule{\linewidth}{0.5mm}}
	
	\center
	
	%------------------------------------------------
	%	Nagłówki
	%------------------------------------------------
	
	\textsc{\LARGE \textbf{Uniwersytet Przyrodniczy we Wrocławiu}}\\[1cm] 
	
	\textsc{\Large Wydział Biologii i Hodowli Zwierząt}\\[0.5cm] 
	
	\textsc{\large Kierunek: Bioinformatyka}\\[0.5cm] 
	
	\textsc{\large Specjalność: Biostatystyka i programowanie bioinformatyczne}\\[0.5cm]
	
	\textsc{\large Przedmiot: Zaawansowane elementy stosowania pakietów statystycznych}\\[0.5cm]
	
	%------------------------------------------------
	%	Tytuł
	%------------------------------------------------
	
	\HRule\\[0.4cm]
	
	{\huge\bfseries Analiza oraz modelowanie łącznej kwoty odrzuconych pożyczek na podstawie raportu Lending Club w latach 2007-2012}\\[0.4cm] 
	
	\HRule\\[1.5cm]
	
	%------------------------------------------------
	%	Autor
	%------------------------------------------------
	
	\begin{minipage}{0.4\textwidth}
		\begin{flushleft}
			\large
			\textit{Autor}\\
			Artur \textsc{Wójtowicz}
		\end{flushleft}
	\end{minipage}
	~
	\begin{minipage}{0.4\textwidth}
		\begin{flushright}
			\large
			\textit{Prowadzący}\\
			dr inż. Anna \textsc{Mucha} % Prowadzący
		\end{flushright}
	\end{minipage}
	
	
	\vfill\vfill\vfill
	{\large\today}
	\vfill 
	
\end{titlepage}

%----------------------------------------------------------------------------------------
\newpage

\tableofcontents

\newpage

\begin{abstract}

\end{abstract}

\newpage

\section{Zbiór danych}

Dane, które zostały użyte w tym sprawozdaniu, pochodzą z głównej strony firmy \textit{Lending Club} - firmy udzielającej pożyczek na terenie USA, zaś dane, są ogólnodostępne pod adresem \hyperref[https://www.lendingclub.com/info/download-data.action]{https://www.lendingclub.com/info/download-data.action}. Zawierają one ponad 755 tysięcy rekordów odrzuconych pożyczek w latach od 26 maja 2007 do 31 grudnia 2012 oraz dziewięć kolumn, na które składają się:
\begin{itemize}
\item Amount Requested - Kwota całkowita wnioskowana przez pożyczkobiorcę;
\item Application Date - Data aplikacji o pożyczkę przez pożyczkobiorcę;
\item Loan Title - Tytuł pożyczki otrzymana przez pożyczkobiorcę;
\item Risk Score - Do 5 listopada 2013 używano tutaj \textit{FICO score} stworzonego przez \textit{Fair Isaac Corporation} do oceny ryzyka udzielenia pożyczki;
\item Debt-To-Income Ratio - Obliczona proporcja względem całkowitych miesięcznych spłat zadłużeń pożyczkodawcy, odliczając od tego zakwaterowanie i wnioskowaną pożyczkę. Wartość ta, jest dzielona przez miesięczne przychody pożyczkobiorcy, które zgłosił;
\item Zip Code - Pierwsze trzy numery kodu pocztowego pożyczkobiorcy;
\item State - Stan, w którym pożyczkodawca składa aplikację o pożyczkę;
\item Employment Length - Okres w latach zatrudnienia. Możliwe wartości znajdują się pomiędzy 0 a 10, gdzie 0 oznacza mniej niż rok zaś 10 więcej niż dziesięc lat.
\item Policy Code - 
\end{itemize}

\clearpage

\listoffigures

\clearpage

\listoftables

\end{document}
